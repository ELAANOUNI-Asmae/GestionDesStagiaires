\documentclass{report}
\usepackage[a4paper, top=2cm, bottom=2cm, left=2.5cm, right=2.5cm]{geometry}
\usepackage[utf8]{inputenc}
\usepackage[T1]{fontenc}
\usepackage[french]{babel}
\usepackage{lmodern}
\usepackage[hyphens]{url}
\usepackage[pdfauthor = {{Asmae ELAANOUNI}}, pdftitle = {{devoir1}}, pdfstartview = Fit, pdfpagelayout = SinglePage, pdfnewwindow = true, bookmarksnumbered = true, breaklinks, colorlinks, linkcolor = black, urlcolor = black, citecolor = black, linktoc = all]{hyperref}
\usepackage{graphicx}
\usepackage{enumitem, pifont}
\usepackage{wrapfig}
\usepackage{float}
\usepackage{xcolor}
\usepackage{tabularx}
\usepackage{enumitem, pifont}
\setlist[itemize, 1]{label ={--}, itemsep =\baselineskip}
\setlist[enumerate, 1]{label= \arabic*), itemsep=\baselineskip}
\renewcommand{\arraystretch}{1}
\usepackage{tocloft}
\begin{document}

\linespread{1}


\begin{minipage}{0.5\linewidth}
  \raggedright
\includegraphics[width=0.6\linewidth]{yool.png}
\end{minipage}
\hfill
\begin{minipage}{0.5\linewidth}
  \raggedleft
\includegraphics[width=0.65\linewidth]{logo.png}
\end{minipage}
\\


\vspace{6cm}

\newcommand{\HRule}{\rule{\linewidth}{0.55mm}}

\begin{center}
\HRule \\[0.25cm]
\huge\textbf{\underline{CdC: Plateforme de Gestion des Stagiaires}} \\[0.25cm]
\HRule \\
 [6cm]
 \end{center}
 \begin{center}
   \Large\textbf{Version :} 1.0\\
   \Large\textbf{Date :} Le \today\\
   \Large\textbf{Auteur :} ELAANOUNI Asmae\\
   \Large\textbf{Destinataires :} Équipe technique de Yool Education\\
 \end{center}

\clearpage
\section*{Introduction}
\subsection*{Contexte}
La plateforme vise à simplifier la gestion administrative des stagiaires (inscription, suivi, évaluation) pour les responsables administratifs.
\subsection*{Objectifs}
\begin{itemize}
  \item Centraliser les données des stagiaires.
  \item Automatiser les tâches administratives (attestations, rapports).
  \item Faciliter le suivi par les administrateurs.
\end{itemize}
\subsection*{Portée}
\begin{itemize}
  \item Inclus : Back-office administratif .
  \item Exclu : Fonctionnalités pour les stagiaires/tuteurs (connexion, upload de documents).
\end{itemize}
\section*{Besoins Fonctionnels (Back-Office Administratif)}
\subsection*{Gestion des stagiaires}
\begin{itemize}
  \item Ajout/Modification/Suppression des profils stagiaires (nom, prénom, formation, dates, encadrant académique/entreprise, Email, Téléphone (optionnel), ,type du stage, sujet de stage ).
  \item Import/Export des données (Excel/CSV).
  \item Recherche et filtres (par formation, statut, date).
\end{itemize}
\subsection*{Suivi administratif}
\begin{itemize}
  \item Génération automatique d’attestations de stage (modèles personnalisables).
  \item Gestion des absences :\\
    - Interface pour saisir les jours d’absence (date + motif).\\
    - Calcul automatique du nombre total d’absences.\\
    - Alertes si dépassement d’un seuil (ex : > 3 jours non justifiés).
  \item Évaluation du stagiaire :
    - Formulaire d’évaluation (note/commentaires) par l’administrateur.\\
    - Grille de critères prédéfinis (ex : ponctualité, qualité du travail).\\
    - Export PDF du bilan d’évaluation.
  \item Historique des actions (logs des modifications).
  \item Tableau de bord (stats : nombre de stagiaires par formation, etc.).
  \item Système d’alerte automatique :\\
    - Si un document manque → **notification visuelle** (badge rouge, icône warning).\\
    - **Email automatique** à l’administrateur (ou liste de destinataires) en cas de dossier incomplet.\\
    - **Rappel quotidien ou hebdomadaire** (paramétrable) tant que le dossier n’est pas complet.
\end{itemize}
\subsection*{Sécurité et accès}
\begin{itemize}
  \item Espace sécurisé (login/mot de passe pour les admins).
  \item Rôles utilisateurs (ex : Admin full access / Lecteur seul).
\end{itemize}
\section*{Besoins Techniques}
\subsection*{Front-End}
\begin{itemize}
  \item Technologies : HTML5, CSS3, JavaScript (Vanilla ou framework léger comme Alpine.js).
  \item Responsive : Compatible bureau/tablette.
  \item Navigateurs supportés : Chrome, Firefox, Edge (dernières versions).
\end{itemize}
\subsection*{Back-End}
\begin{itemize}
  \item Langage : PHP (Framework Laravel).
  \item Base de données : MySQL (structure optimisée pour les jointures).
  \item Sécurité : Protection contre les injections SQL, hachage des mots de passe.
\end{itemize}
\subsection*{Hébergement}
\begin{itemize}
  \item Serveur compatible PHP/MySQL (ex : Apache/Nginx).
\end{itemize}
\section*{Contraintes}
\noindent- **Délai** : Livraison en 4 semaines/mois.\\
- **RGPD** : Confidentialité des données stagiaires (anonymisation possible).\\
\section*{ Livrables Attendus}
\noindent- **Code source** (front + back) commenté.\\
- **Base de données** (script SQL de création).\\
- **Documentation technique** (installation, architecture).\\
\section*{Organisation}
\noindent- **Méthodologie** : Développement agile (sprints avec livraisons partielles).\\
- **Points de suivi** : Réunions hebdomadaires.
\section*{Critères d’Acceptation}
\noindent- **Tests fonctionnels** : Vérification de l’ajout/génération d’attestations.\\
- **Performance** : Temps de réponse < 1s pour les requêtes simples.\\
- **Sécurité** : Audit des vulnérabilités (ex : OWASP).
\end{document}
